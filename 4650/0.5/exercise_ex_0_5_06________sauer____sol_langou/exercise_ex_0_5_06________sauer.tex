
\documentclass[pdftex,11pt]{article}
\usepackage{url}
\usepackage{graphicx}
\usepackage{hyperref}
\usepackage{latexsym,amssymb}
\usepackage{amssymb,amsmath}
\usepackage{color}
\input{../../vrac/rgb.tex}


\usepackage[english]{babel}
\usepackage{array}
\usepackage{multirow} 

\hypersetup{
pdftitle={Langou :: Sauer EX.0.5.6 (answer)},
pdfauthor={Julien Langou}, 
} 

\setlength{\oddsidemargin}{-0.5in}
\setlength{\evensidemargin}{-0.5in}

\setlength{\textwidth}{7.4in}
\setlength{\textheight}{10.0in}

\setlength{\topmargin}{-0.75in}
\setlength{\headheight}{0pt}
\setlength{\headsep}{0pt}

\setlength{\parindent}{0pt}

\begin{document}

\thispagestyle{empty}
\pagestyle{empty}
\renewcommand{\theenumi}{\alph{enumi}}


{\color{blue} Note: In this exercise, we pretty much do not consider round-off
errors.  For example, we will assume that the values returned by matlab when
evaluating $f(x)=x^{-2}$ are exact.  This is justified because our
approximation errors (using Taylor polynomials) are much larger than the
round-off errors. We will see some limitations (when round-off errors are not
negligible any longer) at the end of the exercise.  } \\



%
\framebox{
\begin{minipage}{\textwidth}
{\tiny
{\bf Copyright (C) 2018, 2012, 2016 by Pearson Education Inc. All Rights Reserved,}
please visit \url{www.pearsoned.com/permissions/}.}
%{\bf Copyright (C) 2018, 2012, 2016 by Pearson Education Inc. All Rights Reserved.}
%Printed in the United States of America.
%This publication is protected by copyright, and permission should be obatined from the 
%publisher prior to any prohibted reproduction, storage in retrieval system, or 
%transmission in any form of by any means, electronic, mechanical, photocopying, recording, 
%or otherwise. For information regarding permissions, request forms and the appropriate 
%contacts within the Pearson Education Global Rights \& Permissions department, 
%please visit \url{www.pearsoned.com/permissions/}.}
%
\end{minipage}}



\framebox{
\begin{minipage}{\textwidth}
\textbf{EX.0.5.6, Sauer}\\\\

\begin{enumerate}
\item Find the Taylor polynomial of degree 4 for $f (x) = x^{-2}$ about the point $x_0 = 1.$
\item Use the result of (a) to approximate $f(0.9)$ and $f(1.1)$.\\
\item
Use the Taylor remainder to find an error formula for the Taylor polynomial.
Give error bounds for each of the two approximations made in part (b). Which of
the two approximations in part (b) do you expect to be closer to the correct
value?
\item
Use a calculator to compare the actual error in each case with your error bound from part (c).\\
\end{enumerate}

\end{minipage}}

\end{document}
