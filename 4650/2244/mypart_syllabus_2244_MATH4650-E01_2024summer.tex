
\documentclass[letterpaper,10pt]{article}

\usepackage[margin=.8in]{geometry}
\parindent=0in
%\parskip=1em

\usepackage[compact]{titlesec}  % to reduce spacing around section headings
\usepackage{enumitem}  % allows for compact lists
\usepackage[vskip=0.5em,leftmargin=1.5em]{quoting}  % indented offset text, for e.g. course descriptions

\usepackage{fourier}

\usepackage{times}

\usepackage{color}
\definecolor{darkblue}{rgb}{0,0,0.5}
\usepackage[pdftitle={syllabus for Numerical Analysis I, Summer 2024, Langou},
            colorlinks=true,allcolors=darkblue,linkcolor=darkblue,urlcolor=darkblue]{hyperref}
\renewcommand\UrlFont{\color{darkblue}}  % styling for \url
\newcommand{\mailto}[1]{\href{mailto:#1}{#1}}

\setlength{\parindent}{0em}

\pagestyle{empty}  % no page numbers

\begin{document}

\begin{center}{\LARGE
MATH 4650-E01 -- Numerical Analysis I
}
\bigskip

Summer 2024

%\emph{Class meetings}: Monday--Wednesday 9:30am--10:45am via Zoom

Department of Mathematical and Statistical Sciences, University of Colorado Denver
\end{center}\vspace*{-1em}


%%%%%%%%%%%%%%%%%%%%%%%%%%%%%%%%%%%%%%%%%%%%%%%%%%%%%%%%%%%%%%%%%%%%%%%%%%%%%
\section*{Instructor information}
%%%%%%%%%%%%%%%%%%%%%%%%%%%%%%%%%%%%%%%%%%%%%%%%%%%%%%%%%%%%%%%%%%%%%%%%%%%%%

\emph{Instructor:} Julien Langou,
email: \mailto{julien.langou@ucdenver.edu} or through Canvas.\\

\emph{Office hours:} By appointment. I will try to be as
flexible as possible while trying to maximize the attendance at my office hour.\\

\medskip

%%%%%%%%%%%%%%%%%%%%%%%%%%%%%%%%%%%%%%%%%%%%%%%%%%%%%%%%%%%%%%%%%%%%%%%%%%%%%
\section*{Course catalog description and requisites}
%%%%%%%%%%%%%%%%%%%%%%%%%%%%%%%%%%%%%%%%%%%%%%%%%%%%%%%%%%%%%%%%%%%%%%%%%%%%%

%\begin{quoting}

\emph{MATH 4650 -  Numerical Analysis I - 3 Credits}\\
\\
A first semester course in numerical methods and analysis fundamental to many
algorithms encountered in scientific computing, data science, machine learning,
and computational models in science and engineering. Rounding errors and
numerical stability of algorithms; solution of linear and nonlinear equations;
data modeling with interpolation and least-squares; and optimization methods.\\
\\
This course assumes that students have the equivalent of differential and
integral calculus (e.g., MATH 2411), linear algebra (e.g., MATH 3191 or 3195),
and computer programming (e.g., MATH 1376 or CSCI 1410).\\
\\
Prereq: MATH 3191 or MATH 3195 with a C- or higher.\\ 
\\

%\end{quoting}

%%%%%%%%%%%%%%%%%%%%%%%%%%%%%%%%%%%%%%%%%%%%%%%%%%%%%%%%%%%%%%%%%%%%%%%%%%%%%
\section*{Course Goals}
%%%%%%%%%%%%%%%%%%%%%%%%%%%%%%%%%%%%%%%%%%%%%%%%%%%%%%%%%%%%%%%%%%%%%%%%%%%%%

Completion of this course will provide you with
\begin{enumerate}

\item an understanding of the basic theory of solving mathematical problems
with computers while being cognizant of floating-point arithmetic including
issues of overflow and underflow.

\item knowledge of the different issues surrounding errors in using numerical
methods including machine epsilon, error analysis, convergence, rounding error,
truncation error, and norms.

\item an appreciation of the difficulties involved in finding reliable
solutions as well as be able to apply various methods for estimating errors in
solutions in order to judge how reliable those solutions are.

\item an awareness of conditioning of problems and stability of algorithms and
the distinguish between the two.

\end{enumerate}

\section*{Programming Language}

We will be using Python through Google Colab Jupyter Notebook (\url{https://colab.research.google.com/}).

% This is an upper division level mathematics class and some level of abstraction
% and proof writing and proof skills will be taught and evaluated. This class
% will cover ``matrix computation''/``matrix methods''/``matrix algorithms'' and,
% in addition, some more theoretical concepts. A previous ``proof class'' is not
% required for this class. This class can be viewed as a gentle introduction to
% proof and abstract mathematics.  I will also demonstrate some programming using
% Python and Google Colab (\url{https://colab.research.google.com/}). No
% programming experience is needed.  Similarly as proof, this class can be viewed
% as a gentle introduction to programming.

%%%%%%%%%%%%%%%%%%%%%%%%%%%%%%%%%%%%%%%%%%%%%%%%%%%%%%%%%%%%%%%%%%%%%%%%%%%%%
\section*{Course materials and procedures}
%%%%%%%%%%%%%%%%%%%%%%%%%%%%%%%%%%%%%%%%%%%%%%%%%%%%%%%%%%%%%%%%%%%%%%%%%%%%%

%\hangindent=1em
\emph{Textbook.} \emph{Numerical Analysis, 3rd edition}
by 
\href{http://math.gmu.edu/~tsauer/}{Tim Sauer}, 
ISBN-13: 9780134697376, published by Pearson, in 2018.
The link to the textbook is \url{https://www.pearson.com/en-us/subject-catalog/p/numerical-analysis/P200000006340/9780134697376}.

%\hangindent=1em
%\emph{Procedure for Exam Proctoring.} 
%For exams, I will offer a few Zoom proctoring sessions, and you can pick the
%session that fit your schedule.  In general, I found that Friday works well for
%exams. During the exam, I will welcome everyone to the Zoom session, camera
%must be on, I will send the exam to everyone present, and I expect the exam
%returned on Canvas before you leave the Zoom session.  There will be three
%exams and one final exam.  I realize that this process of proctoring exams
%leaves a lots of loopholes for possible academic dishonesty.  A large part of
%my teaching philosophy is based on trusting you for engaging in academic
%honesty during the semester including during exams. See also the later
%discussion on academic honesty.\\
%
%\hangindent=1em
%\emph{Mandatory virtual meetings.} 
%Although this class is online, in order to get to know you and work together, I
%will require some mandatory virtual meetings as part of your written homework
%assignments. They will be called ``oral exams'', and we will go over some
%written homework questions together in a virtual settings. It is very important
%to me to connect with you (virtually) and so {\color{red} these virtual meetings are a
%requirement for passing this class.}

%\hangindent=1em
\emph{More procedures.} 
Assignments and additional course materials will be posted on Canvas.
Announcements, including any revisions to this syllabus, will be announced and
posted on Canvas.

%%%%%%%%%%%%%%%%%%%%%%%%%%%%%%%%%%%%%%%%%%%%%%%%%%%%%%%%%%%%%%%%%%%%%%%%%%%%%
\section*{Evaluation}
%%%%%%%%%%%%%%%%%%%%%%%%%%%%%%%%%%%%%%%%%%%%%%%%%%%%%%%%%%%%%%%%%%%%%%%%%%%%%

During the semester, grades will be posted on Canvas.
Students should regularly check their recorded grades, and immediately bring
any discrepancies or disputes to the attention of the instructor.  Note that
the grade calculation capabilities of Canvas are limited and may not be
accurate. Use the grading scheme as described in this syllabus to compute your
course grade.\\

You are encouraged to work together in groups outside of class, as well as
consult other resources.  However, your solutions must be your own.  Any
submitted solutions that I feel have been mostly copied from other sources,
including classmates, textbooks, or the web, will receive no credit.  See also
the later discussion on academic honesty.\\

\paragraph{Homework (10\% $\times$ 7 = 70\%).}
There will be 7 homework during the semester. Each homework is worth 10\% of
the final grade.  For each homework, you need to turn in 
a ``Google Colab Jupyter Notebook shared link''
%a ``.IPYNB'' file, 
and a PDF with handwritten or typed solutions.
The homework will be made with some questions from the textbook, some ``reality check'' from the textbook, and/or some questions written by your instructors.
\\

\paragraph{Project (10\% $\times$ 3 = 30\%).}
There will be 3 ``reality check'' projects during the semester. Each project is worth 10\% of
the final grade.
For each reality check, you need to turn in 
a ``Google Colab Jupyter Notebook shared link''
%a ``.IPYNB'' file, 
and a PDF with handwritten or typed solutions.\\

\paragraph{Final course grade scale.}
Final course letter grades will be assigned according to a student's total course score (calculated as described above).
Letter grades for specific scores are given in the following table.
\smallskip

\renewcommand{\arraystretch}{1.15}
\hspace*{3em}\begin{tabular}{rl@{\quad}|@{\quad}rl@{\quad}|@{\quad}rl|@{\quad}rl|@{\quad}rl}
$\ge 93\%$ & A  & $\ge 87\%$ & B+ & $\ge 77\%$ & C+ & $\ge 60\%$ & D\phantom{+} & $< 60\%$ & F\\
$\ge 90\%$ & A- & $\ge 83\%$ & B  & $\ge 70\%$ & C  \\
           &    & $\ge 80\%$ & B- & 
\end{tabular}

\paragraph{Due Dates.}
~\\
\begin{center}
\begin{tabular}{lll}
Homework \#1 & Monday June 10th 2024 & at 11:59pm\\
Homework \#2 & Monday June 17th 2024 & at 11:59pm\\
Homework \#3 & Monday June 24th 2024 & at 11:59pm\\
Homework \#4 & Monday July  1st 2024 & at 11:59pm\\
Homework \#5 & Monday July  8th 2024 & at 11:59pm\\
Homework \#6 & Monday July 15th 2024 & at 11:59pm\\
Homework \#7 & Monday July 22nd 2024 & at 11:59pm\\
Reality Checks \#1, \#2 and \#4  & Friday July 26th 2024 & at 11:59pm\\
\end{tabular}
\end{center}

\end{document}
