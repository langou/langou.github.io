
\documentclass[pdftex,11pt]{article}
\usepackage{url}
\usepackage{graphicx}
\usepackage{hyperref}
\usepackage{latexsym,amssymb}
\usepackage{amssymb,amsmath}
\usepackage{color}
\usepackage{upquote}
\input{../../vrac/rgb.tex}


\usepackage[utf8]{inputenc}

% Default fixed font does not support bold face
\DeclareFixedFont{\ttb}{T1}{txtt}{bx}{n}{12} % for bold
\DeclareFixedFont{\ttm}{T1}{txtt}{m}{n}{12}  % for normal

% Custom colors
\usepackage{color}
\input{../../vrac/rgb.tex}

\usepackage{listings}

\newcommand\digitstyle{\color{deepgreen}}
\makeatletter
\newcommand{\ProcessDigit}[1]
{%
  \ifnum\lst@mode=\lst@Pmode\relax%
   {\digitstyle #1}%
  \else
    #1%
  \fi
}
\makeatother
\lstset{
}


\definecolor{deepblue}{rgb}{0,0,0.5}
\definecolor{deepred}{rgb}{0.6,0,0}
\definecolor{deepgreen}{rgb}{0,0.5,0}
\definecolor{MyLightGray}{rgb}{0.93,0.93,0.93}
\definecolor{MyPurple}{rgb}{.8,0,.8}
\definecolor{MyOrange}{rgb}{.8,0.4,0}


% Python style for highlighting
\newcommand\pythonstyle{\lstset{
literate=
    {0}{{{\ProcessDigit{0}}}}1
    {1}{{{\ProcessDigit{1}}}}1
    {2}{{{\ProcessDigit{2}}}}1
    {3}{{{\ProcessDigit{3}}}}1
    {4}{{{\ProcessDigit{4}}}}1
    {5}{{{\ProcessDigit{5}}}}1
    {6}{{{\ProcessDigit{6}}}}1
    {7}{{{\ProcessDigit{7}}}}1
    {8}{{{\ProcessDigit{8}}}}1
    {9}{{{\ProcessDigit{9}}}}1
    {<=}{{\(\leq\)}}1,
    morestring=[b]",
    morestring=[b]',
    morecomment=[l]//,
backgroundcolor=\color{MyLightGray},
columns=fullflexible
upquote=true,
language=Python,
basicstyle=\ttm,
morekeywords={self},              % Add keywords here
%keywordstyle=\ttb\color{deepblue},
%keywordstyle=\ttb\color{MyOrange},
keywordstyle=\color{MyOrange},
emph={as, import},          % Custom highlighting
%emphstyle=\ttb\color{deepred},    % Custom highlighting style
%emphstyle=\ttb\color{MyPurple},
emphstyle=\color{MyPurple},
stringstyle=\color{deepgreen},
frame=tb,                         % Any extra options here
showstringspaces=false
}}

% Python style for highlighting
\newcommand\outputstyle{\lstset{
%backgroundcolor=\color{MyLightGray},
language=Python,
basicstyle=\ttm,
%morekeywords={self, {1}},              % Add keywords here
%keywordstyle=\ttb\color{deepblue},
%keywordstyle=\ttb\color{MyOrange},
%keywordstyle=\color{MyOrange},
%emph={as, import},          % Custom highlighting
%emphstyle=\ttb\color{deepred},    % Custom highlighting style
%emphstyle=\ttb\color{MyPurple},
%emphstyle=\color{MyPurple},
%stringstyle=\color{deepgreen},
frame=tb,                         % Any extra options here
showstringspaces=false
}}



% Output environment
\lstnewenvironment{pythonoutput}[1][]
{
\outputstyle
\lstset{#1}
}
{}




% Python environment
\lstnewenvironment{python}[1][]
{
\pythonstyle
\lstset{#1}
}
{}

% Python for external files
\newcommand\pythonexternal[2][]{{
\pythonstyle
\lstinputlisting[#1]{#2}}}

% Python for inline
\newcommand\pythoninline[1]{{\pythonstyle\lstinline!#1!}}







\usepackage[english]{babel}
\usepackage{array}
\usepackage{multirow} 

\hypersetup{
pdftitle={Langou :: Sauer EX.0.2.7 (answer)},
pdfauthor={Julien Langou}, 
} 

\setlength{\oddsidemargin}{-0.5in}
\setlength{\evensidemargin}{-0.5in}

\setlength{\textwidth}{7.4in}
\setlength{\textheight}{10.0in}

\setlength{\topmargin}{-0.75in}
\setlength{\headheight}{0pt}
\setlength{\headsep}{0pt}

\setlength{\parindent}{0pt}

\begin{document}

\thispagestyle{empty}
\pagestyle{empty}
\renewcommand{\theenumi}{\alph{enumi}}


%
\framebox{
\begin{minipage}{\textwidth}
{\tiny
{\bf Copyright (C) 2018, 2012, 2016 by Pearson Education Inc. All Rights Reserved,}
please visit \url{www.pearsoned.com/permissions/}.}
%{\bf Copyright (C) 2018, 2012, 2016 by Pearson Education Inc. All Rights Reserved.}
%Printed in the United States of America.
%This publication is protected by copyright, and permission should be obatined from the 
%publisher prior to any prohibted reproduction, storage in retrieval system, or 
%transmission in any form of by any means, electronic, mechanical, photocopying, recording, 
%or otherwise. For information regarding permissions, request forms and the appropriate 
%contacts within the Pearson Education Global Rights \& Permissions department, 
%please visit \url{www.pearsoned.com/permissions/}.}
%
\end{minipage}}




\framebox{
\begin{minipage}{\textwidth}
\textbf{EX.0.2.7, Sauer}\\\\
Convert the following binary numbers to base 10.
(a) $1010101$  
(b) $1011.101$ 
(c) $10111.\overline{01}$
(d) $110.\overline{10}$
(e) $10.\overline{110}$ 
(f) $110.1\overline{101}$
(g) $10.010\overline{1101}$
(h) $111.\overline{1}$. %8
\end{minipage}}


\vspace*{1cm}

\framebox{
\begin{minipage}{\textwidth}
{\tiny
{\bf Copyright (c) 2021, Julien Langou. All rights reserved,}
please visit \url{https://creativecommons.org/licenses/by/4.0/}.}
\end{minipage}}
\vspace*{.2cm}

\textbf{EX.0.2.7, Sauer, solution, Langou}\\


\begin{itemize}

\item Only turning the Python code is not a good answer. 

\item The copy-paste from this PDF to python code does not work great. It is better to copy-paste from colab.

\item The Colab Jupyter Notebook is available at: 
\url{https://colab.research.google.com/drive/1prkH1Saji2f2Th8Lom5_JcOhuuBJQSoY}.
\end{itemize}

\begin{enumerate}
\item
$$ (1010101)_2 = 64+16+4+1 = (85)_{10}$$
$$\color{red} (1010101)_2 =  (85)_{10}$$

\begin{python}
print( 0b1010101 )
print( int( '1010101', 2 ) )
\end{python}
\begin{pythonoutput}
85
85
\end{pythonoutput}


\item
$$(1011.101)_2 = 8+2+1+0.5+0.125 = (11.625)_{10} .$$
$$\color{red}(1011.101)_2 = (11.625)_{10} .$$


\begin{python}
print( ( 0b1011101 ) * ( 2 ** -3 ) )
\end{python}
\begin{pythonoutput}
11.625
\end{pythonoutput}






\item 
$(10111)_2 = 16+4+2+1 = 23$
\begin{eqnarray}
(0.\overline{01})_{2} 
\nonumber & = & \frac{1}{4}+ \frac{1}{16}+ \frac{1}{64}+ \frac{1}{256}+\ldots\\
\nonumber & = & \frac{1}{4}\left(1 + \frac{1}{4}+ \left(\frac{1}{4}\right)^2+ \left(\frac{1}{4}\right)^3+\ldots \right)\\
\nonumber & = & \frac{1}{4}\sum_{n=0}^\infty\left(\frac{1}{4}\right)^n
 = \frac{1}{4}\left(\frac{1}{1-\frac{1}{4}}\right)
 = \frac{1}{4}\frac{4}{3}
 = \frac{1}{3}
\end{eqnarray}



$$\color{red}(10111.\overline{01})_2 = (23\frac{1}{3})_{10} .$$

\begin{python}
# appending 6 times the repetend 01
x = ( 0b10111010101010101 ) * ( 2 ** (-12) )
print( x )
print( 23. + 1./3. )
\end{python}
\begin{pythonoutput}
23.333251953125
23.333333333333332
\end{pythonoutput}

\begin{python}
# appending 30 times the repetend 01
x = 0b10111
y = 0b01
for i in range(0,30):
  y = y * ( 2 ** (-2) )
  x = x + y
print( x )
print( 23. + 1./3. )
\end{python}
\begin{pythonoutput}
23.333333333333332
23.333333333333332
\end{pythonoutput}







\item 
$(110)_2 = 4+2 = 6$
\begin{eqnarray}
(0.\overline{10})_{2} 
\nonumber & = & \frac{1}{2}+ \frac{1}{8}+ \frac{1}{32}+ \frac{1}{128}+\ldots\\
\nonumber & = & \frac{1}{2}\left(1 + \frac{1}{4}+ \left(\frac{1}{4}\right)^2+ \left(\frac{1}{4}\right)^3+\ldots \right)\\
\nonumber & = & \frac{1}{2}\sum_{n=0}^\infty\left(\frac{1}{4}\right)^n
 = \frac{1}{2}\left(\frac{1}{1-\frac{1}{4}}\right)
 = \frac{1}{2}\frac{4}{3}
 = \frac{2}{3}
\end{eqnarray}

$$\color{red}(110.\overline{10})_2 = (6\frac{2}{3})_{10} .$$


\begin{python}
# appending 6 times the repetend 10
x = ( 0b110101010101010 ) * ( 2 ** (-12) )
print( x )
print( 6. + 2./3. )
\end{python}
\begin{pythonoutput}
6.66650390625
6.666666666666667
\end{pythonoutput}

\begin{python}
# appending 30 times the repetend 10
x = 0b110
y = 0b10
for i in range(0,30):
  y = y * ( 2 ** (-2) )
  x = x + y
print( x )
print( 6. + 2./3. )
\end{python}
\begin{pythonoutput}
6.666666666666666
6.666666666666667
\end{pythonoutput}







\item 
$(10)_2 = 2 $
\begin{eqnarray}
(0.\overline{110})_{2} 
\nonumber & = & \left(\frac{1}{2}+ \frac{1}{4}\right)+ \frac{1}{8}\left(\frac{1}{2}+ \frac{1}{4}\right)
+ \left(\frac{1}{8}\right)^2\left(\frac{1}{2}+ \frac{1}{4}\right)
+ \left(\frac{1}{8}\right)^3\left(\frac{1}{2}+ \frac{1}{4}\right)
+ \ldots\\
\nonumber & = & \frac{3}{4}\left(1 + \frac{1}{8}+ \left(\frac{1}{8}\right)^2+ \left(\frac{1}{8}\right)^3+\ldots \right)\\
\nonumber & = & \frac{3}{4}\sum_{n=0}^\infty\left(\frac{1}{8}\right)^n
 = \frac{3}{4}\left(\frac{1}{1-\frac{1}{8}}\right)
 = \frac{3}{4}\frac{8}{7}
 = \frac{6}{7}
\end{eqnarray}

$$\color{red}(10.\overline{110})_2 = (2\frac{6}{7})_{10} .$$



\begin{python}
# appending 5 times the repetend 110
x = ( 0b10110110110110110 ) * ( 2 ** (-15) )
print( x )
print( 2. + 6./7. )
\end{python}
\begin{pythonoutput}
2.85711669921875
2.857142857142857
\end{pythonoutput}

\begin{python}
# appending 20 times the repetend 110
x = 0b10
y = 0b110
for i in range(0,20):
  y = y * ( 2 ** (-3) )
  x = x + y
print( x )
print( 2. + 6./7. )
\end{python}
\begin{pythonoutput}
2.857142857142857
2.857142857142857
\end{pythonoutput}




\item 
$(110)_2 = 6 $
\begin{eqnarray}
(0.1\overline{101})_{2} 
\nonumber & = & 
\frac{1}{2}+
\frac{1}{2}\left(\frac{1}{2}+ \frac{1}{8}\right)
+ \frac{1}{2}\frac{1}{8}\left(\frac{1}{2}+ \frac{1}{8}\right)
+ \frac{1}{2}\left(\frac{1}{8}\right)^2\left(\frac{1}{2}+ \frac{1}{8}\right)
+ \frac{1}{2}\left(\frac{1}{8}\right)^3\left(\frac{1}{2}+ \frac{1}{8}\right)
+ \ldots\\
\nonumber & = & 
\frac{1}{2}+ \frac{5}{16}\left(1 + \frac{1}{8}+ \left(\frac{1}{8}\right)^2+ \left(\frac{1}{8}\right)^3+\ldots \right)\\
\nonumber & = &
\frac{1}{2}+  \frac{5}{16}\sum_{n=0}^\infty\left(\frac{1}{8}\right)^n
 = \frac{1}{2}+  \frac{5}{16}\left(\frac{1}{1-\frac{1}{8}}\right)
 = \frac{1}{2}+  \frac{5}{16}\frac{8}{7}
 = \frac{1}{2}+  \frac{5}{14}
 = \frac{6}{7}
\end{eqnarray}

$$\color{red}(110.1\overline{101})_2 = (6\frac{6}{7})_{10} .$$



\begin{python}
# appending 4 times the repetend 101
x = ( 0b1101101101101101 ) * ( 2 ** (-13) )
print( x )
print( 6. + 6./7. )
\end{python}
\begin{pythonoutput}
6.8570556640625
6.857142857142857
\end{pythonoutput}

\begin{python}
# appending 20 times the repetend 101
x = 0b1101 * ( 2 ** (-1) )
y = 0b101 * ( 2 ** (-1) )
for i in range(0,20):
  y = y * ( 2 ** (-3) )
  x = x + y
print( x )
print( 6. + 6./7. )
\end{python}
\begin{pythonoutput}
2.857142857142857
2.857142857142857
\end{pythonoutput}


\item 
$(10)_2 = 2 $
\begin{eqnarray}
(0.010\overline{1101})_{2} 
\nonumber & = & 
\frac{1}{4}+
\frac{1}{8}\left(\frac{1}{2}+ \frac{1}{4}+\frac{1}{16}\right)
+ \frac{1}{8}\frac{1}{16}\left(\frac{1}{2}+ \frac{1}{4}+\frac{1}{16}\right)
+ \frac{1}{8}\left(\frac{1}{16}\right)^2\left(\frac{1}{2}+ \frac{1}{4}+\frac{1}{16}\right)
+ \ldots\\
\nonumber & = & 
\frac{1}{4}+ \frac{13}{128}\left(1 + \frac{1}{16}+ \left(\frac{1}{16}\right)^2+ \ldots \right)\\
\nonumber & = &
\frac{1}{4}+  \frac{13}{128}\sum_{n=0}^\infty\left(\frac{1}{16}\right)^n
 = \frac{1}{4}+  \frac{13}{128}\left(\frac{1}{1-\frac{1}{16}}\right)
 = \frac{1}{4}+  \frac{13}{128}\frac{16}{15}
 = \frac{1}{4}+  \frac{13}{120}
 = \frac{43}{120}
\end{eqnarray}

$$\color{red}(10.010\overline{1101})_2 = (2\frac{43}{120})_{10} .$$



\begin{python}
# appending 4 times the repetend 1101
x = ( 0b1001011011101 ) * ( 2 ** (-11) )
print( x )
print( 2. + 43./120. )
\end{python}
\begin{pythonoutput}
2.35791015625
2.3583333333333334
\end{pythonoutput}

\begin{python}
# appending 20 times the repetend 1101
x = 0b10010 * ( 2 ** (-3) )
y = 0b1101 * ( 2 ** (-3) )
for i in range(0,20):
  y = y * ( 2 ** (-4) )
  x = x + y
print( x )
print( 2. + 43./120. )
\end{python}
\begin{pythonoutput}
2.3583333333333334
2.3583333333333334
\end{pythonoutput}





\item 
$(111.\overline{1})_2$ is same as $(1000)_2$ so it is $(8)_{10}$.
This is like saying that in base 10 the number $3.9999999\ldots$ is same a $4$.\\

A more rigorous explanation
$(111)_2 = 4+2+1 = 7$
\begin{eqnarray}
(0.\overline{1})_{2} 
\nonumber & = & \frac{1}{2}+ \frac{1}{4}+ \frac{1}{8}+ \frac{1}{16}+\ldots\\
\nonumber & = & \frac{1}{2}\sum_{n=0}^\infty\left(\frac{1}{2}\right)^n
 = \frac{1}{2}\left(\frac{1}{1-\frac{1}{2}}\right)
 = \frac{1}{2}( 2 )
 = 1
\end{eqnarray}
So $\color{red}(111.\overline{1})_2 = 7 + 1 = 8 .$

$$\color{red}(111.\overline{1})_2 = (8)_{10} .$$


\begin{python}
# appending 17 times the repetend 1
x = ( 0b11111111111111111111 ) * ( 2 ** (-17) )
print( x )
\end{python}
\begin{pythonoutput}
7.999992370605469
\end{pythonoutput}

\begin{python}
# appending 60 times the repetend 1
x = 0b111
y = 0b1
for i in range(0,60):
  y = y * ( 2 ** (-1) )
  x = x + y
print( x )
\end{python}
\begin{pythonoutput}
8.0
\end{pythonoutput}

















\end{enumerate}



\end{document}
