
\documentclass[pdftex,11pt]{article}
\usepackage{url}
\usepackage{graphicx}
\usepackage{hyperref}
\usepackage{latexsym,amssymb}
\usepackage{amssymb,amsmath}
\usepackage{color}
\input{../../vrac/rgb}

\usepackage[english]{babel}
\usepackage{array}
\usepackage{multirow} 

\hypersetup{
pdftitle={Langou :: Sauer EX.0.2.5 (answer)},
pdfauthor={Julien Langou}, 
} 

\setlength{\oddsidemargin}{-0.5in}
\setlength{\evensidemargin}{-0.5in}

\setlength{\textwidth}{7.4in}
\setlength{\textheight}{10.0in}

\setlength{\topmargin}{-0.75in}
\setlength{\headheight}{0pt}
\setlength{\headsep}{0pt}

\setlength{\parindent}{0pt}

\begin{document}

\thispagestyle{empty}
\pagestyle{empty}
\renewcommand{\theenumi}{\alph{enumi}}


%
\framebox{
\begin{minipage}{\textwidth}
{\tiny
{\bf Copyright (C) 2018, 2012, 2016 by Pearson Education Inc. All Rights Reserved,}
please visit \url{www.pearsoned.com/permissions/}.}
%{\bf Copyright (C) 2018, 2012, 2016 by Pearson Education Inc. All Rights Reserved.}
%Printed in the United States of America.
%This publication is protected by copyright, and permission should be obatined from the 
%publisher prior to any prohibted reproduction, storage in retrieval system, or 
%transmission in any form of by any means, electronic, mechanical, photocopying, recording, 
%or otherwise. For information regarding permissions, request forms and the appropriate 
%contacts within the Pearson Education Global Rights \& Permissions department, 
%please visit \url{www.pearsoned.com/permissions/}.}
%
\end{minipage}}






\framebox{
\begin{minipage}{\textwidth}
\textbf{EX.0.2.5, Sauer}\\\\

Find the first 15 bits in binary representation of $\pi$.\\
(Hint: taking $\pi$ as 3.1415 or 3.1416 produces the correct 15 first bits.)\\

\end{minipage}}

\vspace*{1cm}

\framebox{
\begin{minipage}{\textwidth}
{\tiny
{\bf Copyright (c) 2021, Julien Langou. All rights reserved,}
please visit \url{https://creativecommons.org/licenses/by/4.0/}.}
\end{minipage}}
\vspace*{.2cm}

\textbf{EX.0.2.5, Sauer, solution, Langou}\\



\vspace*{1cm}


First, 
\[ (3)_{10} = (11)_2 ;\]
second, 
\[ (.1416)_{10} = (.0010010000111....)_2 .\]

The latter because:

\begin{tabular}{r}
 .1416 \\
\hline
0.2832\\
0.5664\\
1.1328\\
0.2656\\
0.5312\\
1.0624\\
0.1248\\
0.2496\\
0.4992\\
0.9984\\
1.9968\\
1.9936\\
1.9872
\end{tabular}

and we read the bits on the leftmost column from top to bottom.

        \[ \pi = (11.0010010000111...)_2 \]

Note: these are the 15 first bits of $\pi$ however the 15 bits number the closest from $\pi$ would be
        \[ (11.0010010001000...)_2 \]





\end{document}
