
\documentclass[pdftex,11pt]{article}
\usepackage{url}
\usepackage{graphicx}
\usepackage{hyperref}
\usepackage{latexsym,amssymb}
\usepackage{amssymb,amsmath}
\usepackage{color}
\input{../../vrac/rgb.tex}

\usepackage[english]{babel}
\usepackage{array}
\usepackage{multirow} 

\hypersetup{
pdftitle={Langou :: Sauer EX.0.2.4 (answer)},
pdfauthor={Julien Langou}, 
} 

\setlength{\oddsidemargin}{-0.5in}
\setlength{\evensidemargin}{-0.5in}

\setlength{\textwidth}{7.4in}
\setlength{\textheight}{10.0in}

\setlength{\topmargin}{-0.75in}
\setlength{\headheight}{0pt}
\setlength{\headsep}{0pt}

\setlength{\parindent}{0pt}

\begin{document}

\thispagestyle{empty}
\pagestyle{empty}
\renewcommand{\theenumi}{\alph{enumi}}


%
\framebox{
\begin{minipage}{\textwidth}
{\tiny
{\bf Copyright (C) 2018, 2012, 2016 by Pearson Education Inc. All Rights Reserved,}
please visit \url{www.pearsoned.com/permissions/}.}
%{\bf Copyright (C) 2018, 2012, 2016 by Pearson Education Inc. All Rights Reserved.}
%Printed in the United States of America.
%This publication is protected by copyright, and permission should be obatined from the 
%publisher prior to any prohibted reproduction, storage in retrieval system, or 
%transmission in any form of by any means, electronic, mechanical, photocopying, recording, 
%or otherwise. For information regarding permissions, request forms and the appropriate 
%contacts within the Pearson Education Global Rights \& Permissions department, 
%please visit \url{www.pearsoned.com/permissions/}.}
%
\end{minipage}}







\framebox{
\begin{minipage}{\textwidth}
\textbf{EX.0.2.4, Sauer}\\\\
Convert the following base 10 numbers to binary.
(a) 11.25 
(b) 2/3 
(c) 3/5 
(d) 3.2 
(e) 30.6 
(f) 99.9.\\
\end{minipage}}



\vspace*{.7cm}

\framebox{
\begin{minipage}{\textwidth}
{\tiny
{\bf Copyright (c) 2021, Julien Langou. All rights reserved,}
please visit \url{https://creativecommons.org/licenses/by/4.0/}.}
\end{minipage}}
\vspace*{.2cm}

\textbf{EX.0.2.4, Sauer, solution, Langou}\\



\begin{enumerate}
\item
Either we write:
$$ 11.25 = 8 + 2 + 1 + \frac{1}{4} = 2^3 + 2^1 + 2^0 + 2^{-2}.$$

Or we use our technique:\\
\begin{minipage}{.4\textwidth}
$$
\begin{array}{rcrcc}
11/2  & = &    5 & R & 1 \\
5/2   & = &    2 & R & 1 \\
2/2  & = &     1 & R & 0 \\
1/2  & = &     0 & R & 1 \\
\end{array}
$$
$$ (11)_{10} = (1011)_2$$
\end{minipage}
\begin{minipage}{.4\textwidth}
$$
\begin{array}{rcrccc}
0.25& *& 2   & = & 0.50 &    \rightarrow     0  \\
0.50& *& 2   & = & 1.00 &    \rightarrow     1  \\
0.00& *& 2   & = & 0.00 &    \rightarrow     0  \\
\hline
0.00 & *& 2   & = & 0.00 &    \rightarrow     0  \\
\end{array}
$$
$$ (0.25)_{10} = (0.01\overline{0})_2$$
\end{minipage}

We find:
$$\color{red} (11.25)_{10} = (1011.01)_2 $$

\item 

$$
\begin{array}{rcrccc}
\frac{2}{3}& *& 2   & = & 1\frac{1}{3}&    \rightarrow     1  \\\\
\frac{1}{3}& *& 2   & = & 0\frac{2}{3}&    \rightarrow     0  \\\\
\hline
\frac{2}{3}& *& 2   & = & 1\frac{1}{3}&    \rightarrow     1  \\\\
\end{array}
$$

We find:
$$\color{red} (\frac{2}{3})_{10} = (0.\overline{10})_2 $$

\underline{Check:}
\begin{eqnarray}
(0.\overline{10})_{2} 
\nonumber & = & \frac{1}{2}+ \frac{1}{8}+ \frac{1}{32}+ \frac{1}{128}+\ldots\\
\nonumber & = & \frac{1}{2}\left(1 + \frac{1}{4}+ \left(\frac{1}{4}\right)^2+ \left(\frac{1}{4}\right)^3+\ldots \right)\\
\nonumber & = & \frac{1}{2}\sum_{n=0}^\infty\left(\frac{1}{4}\right)^n
 = \frac{1}{2}\left(\frac{1}{1-\frac{1}{4}}\right)
 = \frac{1}{2}\frac{4}{3}
 = \frac{2}{3}
\end{eqnarray}

\item
$$
\begin{array}{rcrccc}
.6& *& 2   & = & 1.2&    \rightarrow     1  \\
.2& *& 2   & = & 0.4&    \rightarrow     0  \\
.4& *& 2   & = & 0.8&    \rightarrow     0  \\
.8& *& 2   & = & 1.6&    \rightarrow     1  \\
\hline
.2& *& 2   & = & 0.4&    \rightarrow     1  \\
\end{array}
$$


We find:
$$\color{red} (\frac{3}{5})_{10} = (0.\overline{1001})_2 $$


\underline{Check:}
\begin{eqnarray}
(0.\overline{1001})_{2} 
\nonumber & = & \left(\frac{1}{2}+ \frac{1}{16}\right)+ \frac{1}{16}\left(\frac{1}{2}+ \frac{1}{16}\right)
+ \left(\frac{1}{16}\right)^2\left(\frac{1}{2}+ \frac{1}{16}\right)
+ \left(\frac{1}{16}\right)^3\left(\frac{1}{2}+ \frac{1}{16}\right)
+ \ldots\\
\nonumber & = & \frac{9}{16}\left(1 + \frac{1}{16}+ \left(\frac{1}{16}\right)^2+ \left(\frac{1}{16}\right)^3+\ldots \right)\\
\nonumber & = & \frac{9}{16}\sum_{n=0}^\infty\left(\frac{1}{16}\right)^n
 = \frac{9}{32}\left(\frac{1}{1-\frac{1}{16}}\right)
 = \frac{9}{16}\frac{16}{15}
 = \frac{3}{5}
\end{eqnarray}

\item 
~\\

\begin{minipage}{.4\textwidth}
$$
\begin{array}{rcrcc}
3/2  & = &    1 & R & 1 \\
1/2  & = &    0 & R & 1 \\
\end{array}
$$
$$(3)_{10} = (11)_2$$
\end{minipage}
\begin{minipage}{.4\textwidth}
$$
\begin{array}{rcrccc}
0.2& *& 2   & = & 0.4 &    \rightarrow     0  \\
0.4& *& 2   & = & 0.8 &    \rightarrow     0  \\
0.8& *& 2   & = & 1.6 &    \rightarrow     1  \\
0.6& *& 2   & = & 1.2 &    \rightarrow     1  \\
\hline
0.2& *& 2   & = & 0.4 &    \rightarrow     0  \\
\end{array}
$$
$$ (0.2)_{10} = (0.\overline{0011})_2$$
\end{minipage}

We find:
$$\color{red} (3.2)_{10} = (11.\overline{0011})_2 $$

\item 
~\\

\begin{minipage}{.4\textwidth}
$$
\begin{array}{rcrcc}
30/2  & = &    15 & R & 0 \\
15/2  & = &     7 & R & 1 \\
 7/2  & = &     3 & R & 1 \\
 2/2  & = &     1 & R & 1 \\
 1/2  & = &     0 & R & 1 \\
\end{array}
$$
$$(30)_{10} = (11110)_2$$
(Or we just eyeball that $30=16+8+4+2$.)
\end{minipage}
\begin{minipage}{.4\textwidth}
$$
\begin{array}{rcrccc}
0.6& *& 2   & = & 1.2 &    \rightarrow     1  \\
0.2& *& 2   & = & 0.4 &    \rightarrow     0  \\
0.4& *& 2   & = & 0.8 &    \rightarrow     0  \\
0.8& *& 2   & = & 1.6 &    \rightarrow     1  \\
\hline
0.6& *& 2   & = & 1.2 &    \rightarrow     1  \\
\end{array}
$$
$$ (0.6)_{10} = (0.\overline{1001})_2$$
\end{minipage}

We find:
$$\color{red} (30.6)_{10} = (11110.\overline{1001})_2 $$






\item 
~\\

\begin{minipage}{.4\textwidth}
$$
\begin{array}{rcrcc}
99/2  & = &    49 & R & 1 \\
49/2  & = &    24 & R & 1 \\
24/2  & = &    12 & R & 0 \\
12/2  & = &     6 & R & 0 \\
6/2   & = &     3 & R & 0 \\
3/2   & = &     1 & R & 1 \\
1/2   & = &     0 & R & 1 \\
\end{array}
$$
$$(99)_{10} = (1100011)_2$$
\end{minipage}
\begin{minipage}{.4\textwidth}
$$
\begin{array}{rcrccc}
0.9& *& 2   & = & 1.8 &    \rightarrow     1  \\
0.8& *& 2   & = & 1.6 &    \rightarrow     1  \\
0.6& *& 2   & = & 1.2 &    \rightarrow     1  \\
0.2& *& 2   & = & 0.4 &    \rightarrow     0  \\
0.4& *& 2   & = & 0.8 &    \rightarrow     0  \\
\hline
0.8& *& 2   & = & 1.6 &    \rightarrow     1  \\
\end{array}
$$
$$ (0.9)_{10} = (0.1\overline{1100})_2$$
\end{minipage}

We find:
$$\color{red} (99.9)_{10} = (1100011.1\overline{1100})_2 $$


\end{enumerate}



\end{document}
