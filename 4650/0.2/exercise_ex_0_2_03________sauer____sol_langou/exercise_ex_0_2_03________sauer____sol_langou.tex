
\documentclass[pdftex,11pt]{article}
\usepackage{url}
\usepackage{graphicx}
\usepackage{hyperref}
\usepackage{latexsym,amssymb}
\usepackage{amssymb,amsmath}
\usepackage{color}
\input{../../vrac/rgb.tex}



\usepackage[utf8]{inputenc}

% Default fixed font does not support bold face
\DeclareFixedFont{\ttb}{T1}{txtt}{bx}{n}{12} % for bold
\DeclareFixedFont{\ttm}{T1}{txtt}{m}{n}{12}  % for normal

% Custom colors
\usepackage{color}
\input{../../vrac/rgb.tex}

\usepackage{listings}

\newcommand\digitstyle{\color{deepgreen}}
\makeatletter
\newcommand{\ProcessDigit}[1]
{%
  \ifnum\lst@mode=\lst@Pmode\relax%
   {\digitstyle #1}%
  \else
    #1%
  \fi
}
\makeatother
\lstset{
}


\definecolor{deepblue}{rgb}{0,0,0.5}
\definecolor{deepred}{rgb}{0.6,0,0}
\definecolor{deepgreen}{rgb}{0,0.5,0}
\definecolor{MyLightGray}{rgb}{0.93,0.93,0.93}
\definecolor{MyPurple}{rgb}{.8,0,.8}
\definecolor{MyOrange}{rgb}{.8,0.4,0}


% Python style for highlighting
\newcommand\pythonstyle{\lstset{
literate=
    {0}{{{\ProcessDigit{0}}}}1
    {1}{{{\ProcessDigit{1}}}}1
    {2}{{{\ProcessDigit{2}}}}1
    {3}{{{\ProcessDigit{3}}}}1
    {4}{{{\ProcessDigit{4}}}}1
    {5}{{{\ProcessDigit{5}}}}1
    {6}{{{\ProcessDigit{6}}}}1
    {7}{{{\ProcessDigit{7}}}}1
    {8}{{{\ProcessDigit{8}}}}1
    {9}{{{\ProcessDigit{9}}}}1
    {<=}{{\(\leq\)}}1,
    morestring=[b]",
    morestring=[b]',
    morecomment=[l]//,
backgroundcolor=\color{MyLightGray},
columns=fullflexible
upquote=true,
language=Python,
basicstyle=\ttm,
morekeywords={self},              % Add keywords here
%keywordstyle=\ttb\color{deepblue},
%keywordstyle=\ttb\color{MyOrange},
keywordstyle=\color{MyOrange},
emph={as, import},          % Custom highlighting
%emphstyle=\ttb\color{deepred},    % Custom highlighting style
%emphstyle=\ttb\color{MyPurple},
emphstyle=\color{MyPurple},
stringstyle=\color{deepgreen},
frame=tb,                         % Any extra options here
showstringspaces=false
}}

% Python style for highlighting
\newcommand\outputstyle{\lstset{
%backgroundcolor=\color{MyLightGray},
language=Python,
basicstyle=\ttm,
%morekeywords={self, {1}},              % Add keywords here
%keywordstyle=\ttb\color{deepblue},
%keywordstyle=\ttb\color{MyOrange},
%keywordstyle=\color{MyOrange},
%emph={as, import},          % Custom highlighting
%emphstyle=\ttb\color{deepred},    % Custom highlighting style
%emphstyle=\ttb\color{MyPurple},
%emphstyle=\color{MyPurple},
%stringstyle=\color{deepgreen},
frame=tb,                         % Any extra options here
showstringspaces=false
}}



% Output environment
\lstnewenvironment{pythonoutput}[1][]
{
\outputstyle
\lstset{#1}
}
{}




% Python environment
\lstnewenvironment{python}[1][]
{
\pythonstyle
\lstset{#1}
}
{}

% Python for external files
\newcommand\pythonexternal[2][]{{
\pythonstyle
\lstinputlisting[#1]{#2}}}

% Python for inline
\newcommand\pythoninline[1]{{\pythonstyle\lstinline!#1!}}










\usepackage[english]{babel}
\usepackage{array}
\usepackage{multirow} 

\hypersetup{
pdftitle={Langou :: Sauer EX.0.2.3 (answer)},
pdfauthor={Julien Langou}, 
} 

\setlength{\oddsidemargin}{-0.5in}
\setlength{\evensidemargin}{-0.5in}

\setlength{\textwidth}{7.4in}
\setlength{\textheight}{10.0in}

\setlength{\topmargin}{-0.75in}
\setlength{\headheight}{0pt}
\setlength{\headsep}{0pt}

\setlength{\parindent}{0pt}

\begin{document}

\thispagestyle{empty}
\pagestyle{empty}
\renewcommand{\theenumi}{\alph{enumi}}


%
\framebox{
\begin{minipage}{\textwidth}
{\tiny
{\bf Copyright (C) 2018, 2012, 2016 by Pearson Education Inc. All Rights Reserved,}
please visit \url{www.pearsoned.com/permissions/}.}
%{\bf Copyright (C) 2018, 2012, 2016 by Pearson Education Inc. All Rights Reserved.}
%Printed in the United States of America.
%This publication is protected by copyright, and permission should be obatined from the 
%publisher prior to any prohibted reproduction, storage in retrieval system, or 
%transmission in any form of by any means, electronic, mechanical, photocopying, recording, 
%or otherwise. For information regarding permissions, request forms and the appropriate 
%contacts within the Pearson Education Global Rights \& Permissions department, 
%please visit \url{www.pearsoned.com/permissions/}.}
%
\end{minipage}}







\framebox{
\begin{minipage}{\textwidth}
\textbf{EX.0.2.3, Sauer}\\\\
Convert the following base 10 numbers to binary.
(a) 10.5, (b) 1/3, (c) 5/7,
(d) 12.8, (e) 55.4, (f) 0.1.\\
\end{minipage}}



\vspace*{.7cm}

\framebox{
\begin{minipage}{\textwidth}
{\tiny
{\bf Copyright (c) 2021, Julien Langou. All rights reserved,}
please visit \url{https://creativecommons.org/licenses/by/4.0/}.}
\end{minipage}}
\vspace*{.2cm}

\textbf{EX.0.2.3, Sauer, solution, Langou}\\


\framebox{
\begin{minipage}{\textwidth}

\begin{itemize}

\item Only turning the Python code is not a good answer. 

\item The copy-paste from this PDF to python code does not work great. It is better to copy-paste from colab.

\item The Colab Jupyter Notebook is available at: 
\url{https://colab.research.google.com/drive/15eH4ZJPSJ1K9fCefVra7IKw5Ws36WB6R}.

\item The Python code and its ouput is at the end of this document.

\end{itemize}
\end{minipage}}
\vspace*{.2cm}





\begin{enumerate}
\item \color{red}Convert the base 10 number $(10.5)_{10}$ to binary.\color{black}

Either we write:
$$ 10.5 = 8 + 2 + \frac{1}{2} = 2^3 + 2^1 + 2^{-1}.$$

Or we use our technique:\\
\begin{minipage}{.4\textwidth}
$$
\begin{array}{rcrcc}
10/2  & = &    5 & R & 0 \\
5/2   & = &    2 & R & 1 \\
2/2  & = &     1 & R & 0 \\
1/2  & = &     0 & R & 1 \\
\end{array}
$$
$$ (10)_{10} = (1010)_2$$
\end{minipage}
\begin{minipage}{.4\textwidth}
$$
\begin{array}{rcrccc}
0.50& *& 2   & = & 1.00 &    \rightarrow     1  \\
0.00& *& 2   & = & 0.00 &    \rightarrow     0  \\
\hline
0.00 & *& 2   & = & 0.00 &    \rightarrow     0  \\
\end{array}
$$
$$ (0.5)_{10} = (0.1\overline{0})_2$$
\end{minipage}

We find:
$$\color{red} (10.5)_{10} = (1010.1)_2 $$

\item 

\color{red}Convert the base 10 number $1/3$ to binary.\color{black}
$$
\begin{array}{rcrccc}
\frac{1}{3}& *& 2   & = & 0\frac{2}{3}&    \rightarrow     0  \\\\
\frac{2}{3}& *& 2   & = & 1\frac{1}{3}&    \rightarrow     1  \\\\
\hline
\frac{1}{3}& *& 2   & = & 0\frac{2}{3}&    \rightarrow     0  \\\\
\end{array}
$$

We find:
$$\color{red} (\frac{1}{3})_{10} = (0.\overline{01})_2 $$

\underline{Check:}
\begin{eqnarray}
(0.\overline{01})_{2} 
\nonumber & = & \frac{1}{4}+ \frac{1}{16}+ \frac{1}{64}+ \frac{1}{256}+\ldots\\
\nonumber & = & \frac{1}{4}\left(1 + \frac{1}{4}+ \left(\frac{1}{4}\right)^2+ \left(\frac{1}{4}\right)^3+\ldots \right)\\
\nonumber & = & \frac{1}{4}\sum_{n=0}^\infty\left(\frac{1}{4}\right)^n
 = \frac{1}{4}\left(\frac{1}{1-\frac{1}{4}}\right)
 = \frac{1}{4}\frac{4}{3}
 = \frac{1}{3}
\end{eqnarray}

\item
\color{red}Convert the base 10 number $5/7$ to binary.\color{black}

$$
\begin{array}{rcrccc}
\frac{5}{7}& *& 2   & = & 1\frac{3}{7}&    \rightarrow     1  \\\\
\frac{3}{7}& *& 2   & = & 0\frac{6}{7}&    \rightarrow     0  \\\\
\frac{6}{7}& *& 2   & = & 1\frac{5}{7}&    \rightarrow     1  \\\\
\hline
\frac{5}{7}& *& 2   & = & 1\frac{5}{7}&    \rightarrow     1  \\\\
\end{array}
$$

We find:
$$\color{red} (\frac{5}{7})_{10} = (0.\overline{101})_2 $$


\underline{Check:}
\begin{eqnarray}
(0.\overline{101})_{2} 
\nonumber & = & \frac{5}{8}+ \frac{5}{64}+ \frac{5}{512}+ \frac{5}{4168}+\ldots\\
\nonumber & = & \frac{5}{8}\left(1 + \frac{1}{8}+ \left(\frac{1}{8}\right)^2+ \left(\frac{1}{8}\right)^3+\ldots \right)\\
\nonumber & = & \frac{5}{8}\sum_{n=0}^\infty\left(\frac{1}{8}\right)^n
 = \frac{5}{8}\left(\frac{1}{1-\frac{1}{8}}\right)
 = \frac{5}{8}\frac{8}{7}
 = \frac{5}{7}
\end{eqnarray}












\item 
\color{red}Convert the base 10 number $12.8$ to binary.\color{black}

For the integer part, we can see that $12 = 8 + 4 = 2^3 + 2^2$, so that $(12)_{10}=(1100)_2$.
\begin{minipage}{.4\textwidth}
$$
\begin{array}{rcrcc}
12/2  & = &    6 & R & 0 \\
6/2   & = &    3 & R & 0 \\
3/2  & = &     1 & R & 1 \\
1/2  & = &     0 & R & 1 \\
\end{array}
$$
$$ (12)_{10} = (1100)_2$$
\end{minipage}


Now, for the decimal part,
$$
\begin{array}{rcrccc}
0.8& *& 2   & = & 1.6 &    \rightarrow     1  \\
0.6& *& 2   & = & 1.2 &    \rightarrow     1  \\
0.2& *& 2   & = & 0.4 &    \rightarrow     0  \\
0.4& *& 2   & = & 0.8 &    \rightarrow     0  \\
\hline
0.8& *& 2   & = & 1.6 &    \rightarrow     1  \\
\end{array}
$$
$$ (0.8)_{10} = (0.\overline{1100})_2$$

We find:
$$\color{red} (12.8)_{10} = (1100.\overline{1100})_2 $$

\item 
\color{red}Convert the base 10 number $55.4$ to binary.\color{black}


For the integer part, we can see that $55 = 32 + 16 + 4 + 2 + 1 = 2^5 + 2^4 + 2^2 + 2^1 + 2^0 $, so that $(55)_{10}=(110111)_2$.
\begin{minipage}{.4\textwidth}
$$
\begin{array}{rcrcc}
55/2  & = &   27 & R & 1 \\
27/2   & = &  13 & R & 1 \\
13/2  & = &    6 & R & 1 \\
 6/2  & = &    3 & R & 0 \\
 3/2  & = &    1 & R & 1 \\
 1/2  & = &    0 & R & 1 \\
\end{array}
$$
$$ (55)_{10} = (110111)_2$$
\end{minipage}


Now, for the decimal part,
$$
\begin{array}{rcrccc}
0.4& *& 2   & = & 0.8 &    \rightarrow     0  \\
0.8& *& 2   & = & 1.6 &    \rightarrow     1  \\
0.6& *& 2   & = & 1.2 &    \rightarrow     1  \\
0.2& *& 2   & = & 0.4 &    \rightarrow     0  \\
\hline
0.4& *& 2   & = & 0.8 &    \rightarrow     0  \\
\end{array}
$$
$$ (0.4)_{10} = (0.\overline{0110})_2$$

We find:
$$\color{red} (55.4)_{10} = (110111.\overline{0110})_2 $$

\item 
\color{red}Convert the base 10 number $0.1$ to binary.\color{black}


$$
\begin{array}{rcrccc}
0.1& *& 2   & = & 0.2 &    \rightarrow     0  \\
0.2& *& 2   & = & 0.4 &    \rightarrow     0  \\
0.4& *& 2   & = & 0.8 &    \rightarrow     0  \\
0.8& *& 2   & = & 1.6 &    \rightarrow     1  \\
0.6& *& 2   & = & 1.2 &    \rightarrow     1  \\
\hline
0.2& *& 2   & = & 0.4 &    \rightarrow     0  \\
\end{array}
$$
$$ (0.1)_{10} = (0.0\overline{0011})_2$$

We find:
$$\color{red} (0.1)_{10} = (0.0\overline{0011})_2 $$

\color{blue}
Please note that we can obtain 0.1 by dividing 0.4 by $2^2$ and so indeed 0.1 in base 2 is ``close'' to 0.4 in base 2. 
\color{black}




\end{enumerate}

\begin{python}
# (a) 10.5
x = 0b1010
y = 0b1
z = x + y * 2**(-1)
print( z )
print( 10.5 )
\end{python}
\begin{pythonoutput}
10.5
10.5
\end{pythonoutput}
\begin{python}
# (b) 1/3
x = 0
y = 0b01
for i in range(0,30):
  y = y * ( 2 ** (-2) )
  x = x + y
print( x )
print( 1./3. )
\end{python}
\begin{pythonoutput}
0.3333333333333333
0.3333333333333333
\end{pythonoutput}
\begin{python}
# (c) 5/7
x = 0
y = 0b101
for i in range(0,30):
  y = y * ( 2 ** (-3) )
  x = x + y
print( x )
print( 5./7. )
\end{python}
\begin{pythonoutput}
0.7142857142857142
0.7142857142857143
\end{pythonoutput}
\begin{python}
# (d) 12.8
x = 0b1100
y = 0b1100
for i in range(0,30):
  y = y * ( 2 ** (-4) )
  x = x + y
print( x )
print( 12.8 )
\end{python}
\begin{pythonoutput}
12.8
12.8
\end{pythonoutput}
\begin{python}
# (e) 55.4
x = 0b110111
y = 0b0011
for i in range(0,30):
  y = y * ( 2 ** (-4) )
  x = x + y
print( x )
print( 55.4 )
\end{python}
\begin{pythonoutput}
55.4
55.4
\end{pythonoutput}
\begin{python}
# (f) 0.1
x = 0
y = 0b0011 * ( 2 ** (-1) )
for i in range(0,20):
  y = y * ( 2 ** (-4) )
  x = x + y
print( x )
print( 0.1 )
\end{python}
\begin{pythonoutput}
0.1
0.1
\end{pythonoutput}





\end{document}
