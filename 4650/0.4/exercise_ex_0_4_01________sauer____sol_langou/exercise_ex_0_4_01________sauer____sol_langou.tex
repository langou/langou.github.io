
\documentclass[pdftex,11pt]{article}
\usepackage{url}
\usepackage{graphicx}
\usepackage{hyperref}
\usepackage{latexsym,amssymb}
\usepackage{amssymb,amsmath}
\usepackage{color}
\input{../../vrac/rgb.tex}

\usepackage[english]{babel}
\usepackage{array}
\usepackage{multirow} 

\hypersetup{
pdftitle={Langou :: Sauer EX.0.4.1 (answer)},
pdfauthor={Julien Langou}, 
} 

\setlength{\oddsidemargin}{-0.5in}
\setlength{\evensidemargin}{-0.5in}

\setlength{\textwidth}{7.4in}
\setlength{\textheight}{10.0in}

\setlength{\topmargin}{-0.75in}
\setlength{\headheight}{0pt}
\setlength{\headsep}{0pt}

\setlength{\parindent}{0pt}

\begin{document}

\thispagestyle{empty}
\pagestyle{empty}
\renewcommand{\theenumi}{\alph{enumi}}

%
\framebox{
\begin{minipage}{\textwidth}
{\tiny
{\bf Copyright (C) 2018, 2012, 2016 by Pearson Education Inc. All Rights Reserved,}
please visit \url{www.pearsoned.com/permissions/}.}
%{\bf Copyright (C) 2018, 2012, 2016 by Pearson Education Inc. All Rights Reserved.}
%Printed in the United States of America.
%This publication is protected by copyright, and permission should be obatined from the 
%publisher prior to any prohibted reproduction, storage in retrieval system, or 
%transmission in any form of by any means, electronic, mechanical, photocopying, recording, 
%or otherwise. For information regarding permissions, request forms and the appropriate 
%contacts within the Pearson Education Global Rights \& Permissions department, 
%please visit \url{www.pearsoned.com/permissions/}.}
%
\end{minipage}}






\framebox{
\begin{minipage}{\textwidth}
\textbf{EX.0.4.1, Sauer}\hfill\textbf{Langou}\\\\
Identify for which values of $x$ there is subtraction of nearly equal numbers, and find an
alternate form that avoids the problem.
$$
\textmd{(a)}~~
\frac{1-\sec x}{\tan^2x}\quad
\textmd{(b)}~~
\frac{1-(1-x)^3}{x}\quad
\textmd{(c)}~~
\frac{1}{1+x}-\frac{1}{1-x}
$$
\end{minipage}}



\vspace*{.7cm}

\framebox{
\begin{minipage}{\textwidth}
{\tiny
{\bf Copyright (c) 2021, Julien Langou. All rights reserved,}
please visit \url{https://creativecommons.org/licenses/by/4.0/}.}
\end{minipage}}
\vspace*{.2cm}

\textbf{EX.0.4.1, Sauer, solution, Langou}\\




\begin{enumerate}

\item

We see that there is subtraction of nearly equal numbers whenever $\sec x = 1$,
so whenever $\cos x =1$, so whenever $x$ is near $2k\pi$ for $k\in\mathbb{Z}$.

We remember our trig formula: 
$$\sec^2 = 1+\tan^2x.$$
This is coming from 
$$1+\tan^2x = 1+\frac{\sin^2x}{\cos^2x}=\frac{\cos^2x+\sin^2x}{\cos^2x}=\frac{1}{\cos^2x}=\sec^2x.$$
So in other words, we have that
$$\sec^2-1 = \tan^2x.$$
Once we see this and our function at hand, it makes sense to multiply and divide by $(1-\sec x)$, we get
$$\frac{1-\sec x}{\tan^2x}
=
\left(\frac{1-\sec x}{\tan^2x}\right)
\left(\frac{1+\sec x}{1+\sec x}\right)
=
\left(\frac{1-\sec^2 x}{\tan^2x}\right)
\left(\frac{-1}{1+\sec x}\right)
=
\frac{-1}{1+\sec x}.$$

The $\frac{-1}{1+\sec x}$ alternate form avoids the cancellation problem for $x$ near $2k\pi$ for $k\in\mathbb{Z}$.

\item

We see that there is subtraction of nearly equal numbers whenever $(1-x)^3 = 1$, so whenever $x$ is near $0$.

Some algebra:
$$\frac{1-(1-x)^3}{x} 
= 
\frac{x^3-3x^2+3x}{x} 
=x^2-3x+3.
$$

So an alternate form that avoids the cancellation problem for $x$ near $0$ is $x^2-3x+3$.

\item

We see that there is subtraction of nearly equal numbers whenever $x$ is near $0$.

Some algebra:
$$
\frac{1}{1+x}-\frac{1}{1-x}
= 
\frac{-2x}{(1+x)(1-x)}
=
\frac{2x}{(1+x)(x-1)}
$$

So an alternate form that avoids the cancellation problem for $x$ near $0$ is 
$\frac{2x}{(1+x)(x-1)}$.

\end{enumerate}



\end{document}
