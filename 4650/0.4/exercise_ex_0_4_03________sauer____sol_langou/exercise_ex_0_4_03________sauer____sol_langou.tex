
\documentclass[pdftex,11pt]{article}
\usepackage{comment}
\usepackage{url}
\usepackage{graphicx}
\usepackage{hyperref}
\usepackage{latexsym,amssymb}
\usepackage{amssymb,amsmath}
\usepackage{color}
\input{../../vrac/rgb.tex}





\usepackage[utf8]{inputenc}

% Default fixed font does not support bold face
\DeclareFixedFont{\ttb}{T1}{txtt}{bx}{n}{12} % for bold
\DeclareFixedFont{\ttm}{T1}{txtt}{m}{n}{12}  % for normal

% Custom colors
\usepackage{color}
\input{../../vrac/rgb.tex}

\usepackage{listings}

\newcommand\digitstyle{\color{deepgreen}}
\makeatletter
\newcommand{\ProcessDigit}[1]
{%
  \ifnum\lst@mode=\lst@Pmode\relax%
   {\digitstyle #1}%
  \else
    #1%
  \fi
}
\makeatother
\lstset{
}


\definecolor{deepblue}{rgb}{0,0,0.5}
\definecolor{deepred}{rgb}{0.6,0,0}
\definecolor{deepgreen}{rgb}{0,0.5,0}
\definecolor{MyLightGray}{rgb}{0.93,0.93,0.93}
\definecolor{MyPurple}{rgb}{.8,0,.8}
\definecolor{MyOrange}{rgb}{.8,0.4,0}


% Python style for highlighting
\newcommand\pythonstyle{\lstset{
literate=
    {0}{{{\ProcessDigit{0}}}}1
    {1}{{{\ProcessDigit{1}}}}1
    {2}{{{\ProcessDigit{2}}}}1
    {3}{{{\ProcessDigit{3}}}}1
    {4}{{{\ProcessDigit{4}}}}1
    {5}{{{\ProcessDigit{5}}}}1
    {6}{{{\ProcessDigit{6}}}}1
    {7}{{{\ProcessDigit{7}}}}1
    {8}{{{\ProcessDigit{8}}}}1
    {9}{{{\ProcessDigit{9}}}}1
    {<=}{{\(\leq\)}}1,
    morestring=[b]",
    morestring=[b]',
    morecomment=[l]//,
backgroundcolor=\color{MyLightGray},
columns=fullflexible
upquote=true,
language=Python,
basicstyle=\ttm,
morekeywords={self},              % Add keywords here
%keywordstyle=\ttb\color{deepblue},
%keywordstyle=\ttb\color{MyOrange},
keywordstyle=\color{MyOrange},
emph={as, import},          % Custom highlighting
%emphstyle=\ttb\color{deepred},    % Custom highlighting style
%emphstyle=\ttb\color{MyPurple},
emphstyle=\color{MyPurple},
stringstyle=\color{deepgreen},
frame=tb,                         % Any extra options here
showstringspaces=false
}}

% Python style for highlighting
\newcommand\outputstyle{\lstset{
%backgroundcolor=\color{MyLightGray},
language=Python,
basicstyle=\ttm,
%morekeywords={self, {1}},              % Add keywords here
%keywordstyle=\ttb\color{deepblue},
%keywordstyle=\ttb\color{MyOrange},
%keywordstyle=\color{MyOrange},
%emph={as, import},          % Custom highlighting
%emphstyle=\ttb\color{deepred},    % Custom highlighting style
%emphstyle=\ttb\color{MyPurple},
%emphstyle=\color{MyPurple},
%stringstyle=\color{deepgreen},
frame=tb,                         % Any extra options here
showstringspaces=false
}}



% Output environment
\lstnewenvironment{pythonoutput}[1][]
{
\outputstyle
\lstset{#1}
}
{}




% Python environment
\lstnewenvironment{python}[1][]
{
\pythonstyle
\lstset{#1}
}
{}

% Python for external files
\newcommand\pythonexternal[2][]{{
\pythonstyle
\lstinputlisting[#1]{#2}}}

% Python for inline
\newcommand\pythoninline[1]{{\pythonstyle\lstinline!#1!}}









\usepackage[english]{babel}
\usepackage{array}
\usepackage{multirow} 

\hypersetup{
pdftitle={Langou :: Sauer EX.0.4.3 (answer)},
pdfauthor={Julien Langou}, 
} 

\setlength{\oddsidemargin}{-0.5in}
\setlength{\evensidemargin}{-0.5in}

\setlength{\textwidth}{7.4in}
\setlength{\textheight}{10.0in}

\setlength{\topmargin}{-0.75in}
\setlength{\headheight}{0pt}
\setlength{\headsep}{0pt}

\setlength{\parindent}{0pt}

\begin{document}

\thispagestyle{empty}
\pagestyle{empty}
\renewcommand{\theenumi}{\alph{enumi}}

%
\framebox{
\begin{minipage}{\textwidth}
{\tiny
{\bf Copyright (C) 2018, 2012, 2016 by Pearson Education Inc. All Rights Reserved,}
please visit \url{www.pearsoned.com/permissions/}.}
%{\bf Copyright (C) 2018, 2012, 2016 by Pearson Education Inc. All Rights Reserved.}
%Printed in the United States of America.
%This publication is protected by copyright, and permission should be obatined from the 
%publisher prior to any prohibted reproduction, storage in retrieval system, or 
%transmission in any form of by any means, electronic, mechanical, photocopying, recording, 
%or otherwise. For information regarding permissions, request forms and the appropriate 
%contacts within the Pearson Education Global Rights \& Permissions department, 
%please visit \url{www.pearsoned.com/permissions/}.}
%
\end{minipage}}





\framebox{
\begin{minipage}{\textwidth}
\textbf{EX.0.4.3, Sauer}\\\\
Explain how to most accurately compute the two roots of the equation $x^2 + bx
- 10^{-12} = 0$, where $b$ is a number greater than $100$.\\
\end{minipage}}


\vspace*{.7cm}

\framebox{
\begin{minipage}{\textwidth}
{\tiny
{\bf Copyright (c) 2021, Julien Langou. All rights reserved,}
please visit \url{https://creativecommons.org/licenses/by/4.0/}.}
\end{minipage}}
\vspace*{.2cm}

\textbf{EX.0.4.3, Sauer, solution, Langou}\\


\framebox{
\begin{minipage}{\textwidth}

\begin{itemize}

\item Only turning the Python code is not a good answer. 

\item The copy-paste from this PDF to python code does not work great. It is better to copy-paste from colab.

\item The Colab Jupyter Notebook is available at: 
\url{https://colab.research.google.com/drive/19c-eoifrUWtfhXvYtjqMIsPeFEXW32CP}.

\item The Python code and its ouput is at the end of this document.

\end{itemize}
\end{minipage}}
\vspace*{.2cm}






We can use the standard formulae for the roots for a polynomial of degree 2:
        \begin{eqnarray*}
            x_{1} = \frac{1}{2}(-b + \sqrt{b^2 + 4 \cdot 10^{-12}}) \\
            x_{2} = \frac{1}{2}(-b - \sqrt{b^2 + 4 \cdot 10^{-12}}).
        \end{eqnarray*}
However, for large $b$, say $b$ larger than $100$, we see that the formula for the first root, $x_1$,
we lead to something like this:

When computed $ \Delta = b^2 + 4 \cdot 10^{-12}$ will be approximately $b^2$, then $\sqrt{\Delta}$
will be approximately $b$, and so $x_1$ will be be approximately $b-b$ so $0$.

The issue is that there is no room in the 52-bit mantissa of $\Delta$ to store $4 \cdot 10^{-12}$.
$b$ is about $10^2$, so $b^2$ is about $10^4$, so larger than $2^{13}$. 
Then $4 \cdot 10^{-12}$ is smaller than $2^{-37}$. 
Because the gap between $2^{13}$ and $2^{-37}$ is $2^{50}$, this means that there are at least 50 bits
in the 52-bit representation of $ \Delta = b^2 + 4 \cdot 10^{-12}$ that are zeros, and then you can start
representing $4 \cdot 10^{-12}$, so you have a very poor accuracy.
Even worse, for $b=10^3$, then $4 \cdot 10^{-12}$ is washed away and we actually have 
$$ fl( b^2 + 4 \cdot 10^{-12} ) = b .$$

A better formula when $b$ is larger than $100$, (and $a$ is 1, and $c$ is $-10^{-12}$), for the first root, $x_1$ is
           $$ x_{1} = - \frac{2 \cdot 10^{-12}}{b + \sqrt{b^2 + 4 \cdot 10^{-12}} } $$

Here is a python code to show this.  Here we set $b=10^3$.
We first compute a ``trusted'' answer using
\pythoninline{numpy.root}, and we find that $x_1$ should be around $10^{-15}$.
Then we use the ``bad'' formula, and we find that $x_1 = 0$. Given
that $x_1$ should be around $10^{-15}$, the relative error by returning $x_1 = 0$
infinitely
large. Then we use the ``good'' formula, and we find $x_1 = 10^{-15}$, which is much more in accordance with \pythoninline{numpy.root}.

\begin{python}
from math import sqrt
import numpy as np
\end{python}
\begin{python}
a = 1.
b = 1e3
c = -1e-12
\end{python}
\begin{python}
r = np.roots( np.array( [ a, b, c ]) )
print( r[0], r[1] )
\end{python}
\begin{pythonoutput}
-1000.0 1e-15
\end{pythonoutput}
\begin{python}
x0 = ( - b - sqrt( b**2 - 4 * a * c ) ) / ( 2 * a )
x1 = ( - b + sqrt( b**2 - 4 * a * c ) ) / ( 2 * a )
print( x0, x1 )
\end{python}
\begin{pythonoutput}
-1000.0 0.0
\end{pythonoutput}
\begin{python}
x0 = ( - b - sqrt( b**2 - 4 * a * c ) ) / ( 2 * a )
x1 = - ( 2 * c ) / ( b + sqrt( b**2 + 4 * a * c ) )
print( x0, x1 )
\end{python}
\begin{pythonoutput}
-1000.0 1e-15
\end{pythonoutput}







\end{document}
