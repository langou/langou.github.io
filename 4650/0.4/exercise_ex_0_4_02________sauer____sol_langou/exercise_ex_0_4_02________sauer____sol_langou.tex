
\documentclass[pdftex,11pt]{article}
\usepackage{url}
\usepackage{graphicx}
\usepackage{hyperref}
\usepackage{latexsym,amssymb}
\usepackage{amssymb,amsmath}
\usepackage{color}
\input{../../vrac/rgb.tex}


\usepackage[english]{babel}
\usepackage{array}
\usepackage{multirow} 

\hypersetup{
pdftitle={Langou :: Sauer EX.0.4.2 (answer)},
pdfauthor={Julien Langou}, 
} 

\setlength{\oddsidemargin}{-0.5in}
\setlength{\evensidemargin}{-0.5in}

\setlength{\textwidth}{7.4in}
\setlength{\textheight}{10.0in}

\setlength{\topmargin}{-0.75in}
\setlength{\headheight}{0pt}
\setlength{\headsep}{0pt}

\setlength{\parindent}{0pt}

\begin{document}

\thispagestyle{empty}
\pagestyle{empty}
\renewcommand{\theenumi}{\alph{enumi}}

%
\framebox{
\begin{minipage}{\textwidth}
{\tiny
{\bf Copyright (C) 2018, 2012, 2016 by Pearson Education Inc. All Rights Reserved,}
please visit \url{www.pearsoned.com/permissions/}.}
%{\bf Copyright (C) 2018, 2012, 2016 by Pearson Education Inc. All Rights Reserved.}
%Printed in the United States of America.
%This publication is protected by copyright, and permission should be obatined from the 
%publisher prior to any prohibted reproduction, storage in retrieval system, or 
%transmission in any form of by any means, electronic, mechanical, photocopying, recording, 
%or otherwise. For information regarding permissions, request forms and the appropriate 
%contacts within the Pearson Education Global Rights \& Permissions department, 
%please visit \url{www.pearsoned.com/permissions/}.}
%
\end{minipage}}





\framebox{
\begin{minipage}{\textwidth}
\textbf{EX.0.4.2, Sauer}\\\\
Find the roots of the equation
$x^{2}+3x-8^{-14}=0$ with three-digit accuracy.\\
\end{minipage}}


\vspace*{.7cm}

\framebox{
\begin{minipage}{\textwidth}
{\tiny
{\bf Copyright (c) 2021, Julien Langou. All rights reserved,}
please visit \url{https://creativecommons.org/licenses/by/4.0/}.}
\end{minipage}}
\vspace*{.2cm}

\textbf{EX.0.4.2, Sauer, solution, Langou}\\





        \begin{eqnarray*}
            x_{1} = \frac{1}{2}(-3 + \sqrt{9 + 4 \cdot 8^{-14}}) \\
            x_{2} = \frac{1}{2}(-3 - \sqrt{9 + 4 \cdot 8^{-14}}).
        \end{eqnarray*}
        Using a calculator, we find that $8^{-14} \approx 2.27374 \cdot 10^{-13}$, and can compute directly
        \[x_{2}  \approx -0.5*(3+\sqrt{9+4(2.27374 \cdot 10^{-13})}) \approx -3.000. \]
        This will not work for $x_{1}$, however, since $\sqrt{9+4(2.27374 \cdot 10^{-13})} = 3.000$ to three
        decimal places, which would result in $x_{1}=0$.  If we instead use 
        \begin{eqnarray*} 
            x_{1} &=& \frac{1}{2}(-3 + \sqrt{9 + 4 \cdot 8^{-14}}) = \frac{-9 + 9 + 4 \cdot 8^{-14}}{2(3 + \sqrt{9 + 4 \cdot 8^{-14}})} = \frac{4 \cdot 8^{-14}}{2(3 + \sqrt{9 + 4 \cdot 8^{-14}})}= \frac{2 \cdot 8^{-14}}{3 + \sqrt{9 + 4 \cdot 8^{-14}}} \\
            &\approx& \frac{2 \cdot 2.27374 \cdot 10^{-13}}{6.000} \approx 0.758 \cdot 10^{-14}
        \end{eqnarray*}      



\end{document}
