\documentclass[pdftex,11pt]{article}
\usepackage{url}
\usepackage{graphicx}
\usepackage{hyperref}
\usepackage{latexsym,amssymb}
\usepackage{amssymb,amsmath}
\usepackage{color}
\input{../../vrac/rgb.tex}

\usepackage[english]{babel}
\usepackage{array}
\usepackage{multirow} 

\hypersetup{
pdftitle={Langou :: Sauer EX.0.1.6 (question)},
pdfauthor={Julien Langou}, 
} 

\setlength{\oddsidemargin}{-0.5in}
\setlength{\evensidemargin}{-0.5in}

\setlength{\textwidth}{7.4in}
\setlength{\textheight}{10.0in}

\setlength{\topmargin}{-0.75in}
\setlength{\headheight}{0pt}
\setlength{\headsep}{0pt}

\setlength{\parindent}{0pt}

\begin{document}

\thispagestyle{empty}
\pagestyle{empty}
\renewcommand{\theenumi}{\alph{enumi}}

\framebox{
\begin{minipage}{\textwidth}
{\tiny
{\bf Copyright (c) 2021, Julien Langou. All rights reserved,}
please visit \url{https://creativecommons.org/licenses/by/4.0/}.}
\end{minipage}}
\vspace*{.2cm}


\framebox{
\begin{minipage}{\textwidth}
\textbf{EX.0.1.6b, Langou}\\\\
Explain how to evaluate the polynomial for a given input $x$, using as few operations as possible. 
How many multiplications and how many additions are required?
\begin{enumerate}
\item $ p(x) =  a_0 + a_{11}x^{11} + a_{22}x^{22} + a_{33}x^{33} $
\item $ p(x) =  a_{13}x^{13} + a_{24}x^{24} + a_{35}x^{35} + a_{46}x^{46} + a_{57}x^{57} $
% \item $ p(x) =  a_0 + a_5x^5 + a_{10}x^{10} + a_{15}x^{15} $
% \item $ p(x) =  a_7 x^7 + a_{12} x^{12} + a_{17}x^{17} + a_{22}x^{22} + a_{27}x^{27}$
\end{enumerate}
\end{minipage}}

\vspace*{.7cm}

% \framebox{
% \begin{minipage}{\textwidth}
% {\tiny
% {\bf Copyright (c) 2021, Julien Langou. All rights reserved,}
% please visit \url{https://creativecommons.org/licenses/by/4.0/}.}
% \end{minipage}}
% \vspace*{.2cm}
% \textbf{EX.0.1.1, Sauer, solution, Langou}\\



\end{document}
