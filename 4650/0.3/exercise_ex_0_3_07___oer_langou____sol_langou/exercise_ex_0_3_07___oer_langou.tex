
\documentclass[pdftex,11pt]{article}
\usepackage{url}
\usepackage{graphicx}
\usepackage{hyperref}
\usepackage{latexsym,amssymb}
\usepackage{amssymb,amsmath}
\usepackage{color}
\input{../../vrac/rgb.tex}

\usepackage[english]{babel}
\usepackage{array}
\usepackage{multirow} 

\hypersetup{
pdfauthor={Julien Langou}, 
} 

\setlength{\oddsidemargin}{-0.5in}
\setlength{\evensidemargin}{-0.5in}

\setlength{\textwidth}{7.4in}
\setlength{\textheight}{10.0in}

\setlength{\topmargin}{-0.75in}
\setlength{\headheight}{0pt}
\setlength{\headsep}{0pt}

\setlength{\parindent}{0pt}

\begin{document}

\thispagestyle{empty}
\pagestyle{empty}
\renewcommand{\theenumi}{\alph{enumi}}


\framebox{
\begin{minipage}{\textwidth}
{\tiny
{\bf Copyright (c) 2021, Julien Langou. All rights reserved,}
please visit \url{https://creativecommons.org/licenses/by/4.0/}.}
\end{minipage}}
\vspace*{.2cm}



\framebox{
\begin{minipage}{\textwidth}
\textbf{EX.0.3.7, Langou}\\\\
Write each of the given numbers in Matlab's {\em format hex}.
Show your work. Then check your answers with Matlab. 
(a) 16 (b) 130 (c) 1/4 (d) $\textmd{fl}(1/7)$ (e) $\textmd{fl}(4/7)$ (f) $\textmd{fl}(0.01)$ (g) $\textmd{fl}(-0.01)$
(h) $\textmd{fl}(-0.02)$
\end{minipage}}

\end{document}
