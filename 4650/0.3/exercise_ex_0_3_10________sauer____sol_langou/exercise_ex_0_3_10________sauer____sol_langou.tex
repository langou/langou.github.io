
\documentclass[pdftex,11pt]{article}
\usepackage{url}
\usepackage{graphicx}
\usepackage{hyperref}
\usepackage{latexsym,amssymb}
\usepackage{amssymb,amsmath}
\usepackage{color}
\input{../../vrac/rgb.tex}

\usepackage[english]{babel}
\usepackage{array}
\usepackage{multirow} 

\hypersetup{
pdftitle={Langou :: Sauer EX.0.3.10 (answer)},
pdfauthor={Julien Langou}, 
} 

\setlength{\oddsidemargin}{-0.5in}
\setlength{\evensidemargin}{-0.5in}

\setlength{\textwidth}{7.4in}
\setlength{\textheight}{10.0in}

\setlength{\topmargin}{-0.75in}
\setlength{\headheight}{0pt}
\setlength{\headsep}{0pt}

\setlength{\parindent}{0pt}

\begin{document}

\thispagestyle{empty}
\pagestyle{empty}
\renewcommand{\theenumi}{\alph{enumi}}




%
\framebox{
\begin{minipage}{\textwidth}
{\tiny
{\bf Copyright (C) 2018, 2012, 2016 by Pearson Education Inc. All Rights Reserved,}
please visit \url{www.pearsoned.com/permissions/}.}
%{\bf Copyright (C) 2018, 2012, 2016 by Pearson Education Inc. All Rights Reserved.}
%Printed in the United States of America.
%This publication is protected by copyright, and permission should be obatined from the 
%publisher prior to any prohibted reproduction, storage in retrieval system, or 
%transmission in any form of by any means, electronic, mechanical, photocopying, recording, 
%or otherwise. For information regarding permissions, request forms and the appropriate 
%contacts within the Pearson Education Global Rights \& Permissions department, 
%please visit \url{www.pearsoned.com/permissions/}.}
%
\end{minipage}}



\framebox{
\begin{minipage}{\textwidth}
\textbf{EX.0.3.10, Sauer}\\\\

Decide whether $\mathrm{fl}(1+x)>1$ in double precision floating point arithmetic,
        with Rounding to Nearest.  
        \begin{enumerate}
            \item $x=2^{-53}$
            \item $x=2^{-53}+2^{-60}$
        \end{enumerate}

\end{minipage}}



\vspace*{.7cm}

\framebox{
\begin{minipage}{\textwidth}
{\tiny
{\bf Copyright (c) 2021, Julien Langou. All rights reserved,}
please visit \url{https://creativecommons.org/licenses/by/4.0/}.}
\end{minipage}}
\vspace*{.2cm}

\textbf{EX.0.3.10, Sauer, solution, Langou}\\

For double precision floating point, the machine epsilon is $\epsilon =
2^{-52}$.  In (a), $x=2^{-53}$, and thus $x+1$ is an exceptional case for IEEE
floating point rounding. $1+2^{-53}$ is in the exact middle of its two nearest
floating point numbers $x_\textmd{min}=1$ and $x_\textmd{max}=1+2^{-52}$.
In this case, this is a draw and  $x_\textmd{min}=1$ and $x_\textmd{max}=1+2^{-52}$ are equally near.
The rule tells us that we should pick the one which 52nd bit is a 0. Since
\begin{eqnarray}
\nonumber x_\textmd{min}=1  &=&  (1.0000000000000000000000000000000000000000000000000000)_2 \\
\nonumber x_\textmd{max}=1+2^{-52} &=& (1.0000000000000000000000000000000000000000000000000001)_2 
\end{eqnarray}
we have 
\[ \mathrm{fl}(1+2^{-53}) = 1, \] 
thus
$\mathrm{fl}(1+x)>1$ is false for (a).

In (b), 
$$ x_\textmd{min}= 1 < 1+ 2^{-53}+2^{-60} < x_\textmd{max}=1+2^{-52}. $$
However $x_\textmd{max}$ is closer from $1+x$ than  $x_\textmd{min}$ so
$$\mathrm{fl}(1+x)=x_\textmd{max}=1+2^{-52},$$
and
thus $\mathrm{fl}(1+x)>1$ is true for part (b).




\end{document}
