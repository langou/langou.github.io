\documentclass[11pt,letterpaper]{article}
\begin{document}

\begin{center}
{\Large \bf Numerical Linear Algebra }\\
{\Large \bf Weekly assignment \#1 }
\end{center}

\begin{enumerate}

\item
Write a Julia code that creates a random 500x500 system of linear
equations. Solve it using the built-in LU with partial pivoting factorization, 
\texttt{L, U, p = lu(A)}.
Check that the backward error is small.
Report the performance in GFlop/sec.

You can do neither do
\texttt{F = lu(A); x = F \ b} nor 
\texttt{x = A \ b}. You need to directly use the factors L, U and the permutation vector p.


\item
Write a Julia code that creates a random 500x500 
symmetric positive definite system of linear
equations. Solve it using the built-in Cholesky factorization, 
\texttt{L = cholesky(A).L}.
Check that the backward error is small.
Report the performance in GFlop/sec.

You can do neither do
\texttt{F = cholesky(A); x = F \ b} nor 
\texttt{x = A \ b}. You need to directly use the factor L.



\item
Write a Julia code for a recursive Cholesky factorization.
Check that the backward error is small.
Report the performance in GFlop/sec.

\item
Write a Julia code that creates a random 500x500
nonsymmetric system of linear
equations with prescribed condition number $\kappa$.
Check that 
(1) 
$\textmd{backward error} \approx \textmd{machine precision}$
and that
(2)
$  \textmd{forward error} \approx \textmd{condition number} \times \textmd{backward error}. $

\item
Write a Julia code that performs LU with complete pivoting.
Write the test code that goes with it.

\item
Using the already written codes in the textbook compare the 
backward error for a random 500x500 nonsymmetric system of linear
equations for 
LU with no pivoting, 
LU with partial pivoting, 
LU with rook pivoting and 
LU with complete pivoting.

\end{enumerate}



\end{document}




\end{document}
