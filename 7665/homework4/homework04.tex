\documentclass[11pt,letterpaper]{article}
\usepackage{amssymb}

\begin{document}

\begin{center}
{\Large \bf Weekly assignment \#4 }
\end{center}

\small

\begin{enumerate}

\item 
(julia only)
Take a random $6$-by-$6$ matrix $A$. 
\begin{verbatim}
   n = 6
   A = rand(rng, n, n)
\end{verbatim}
We compute a Hessenberg reduction of $A$ with
\begin{verbatim}
   H = copy(A)
   H, tau = LAPACK.gehrd!(H)
   Q = tril(H,-2) 
   Q = Q[2:n,1:n-1]
   LAPACK.orgqr!(Q,tau)
   Q = [ 1 zeros(1,n-1); zeros(n-1,1) Q]
   H = triu(H,-1)
## check that
   @show opnorm( Q*H*Q'-A, 1) / opnorm( A, 1)
   @show opnorm( Q*Q' - I(n), 1)
\end{verbatim}

Let $q_1$ defined as
$$
   q1 = \left(\begin{array}{cccccc} 0.3 & 0.4 & 0.5 & -0.5 & -0.4 & -0.3\end{array}\right)^T.
$$
Observe that $\|q_1\|_2 = 1$.
\begin{verbatim}
   q1 = [ 0.3 0.4 0.5 -0.5 -0.4 -0.3]' 
   @show norm(q1)
\end{verbatim}

\underline{Question:} Reduce $A$ to Hessenberg form such that the first column of the $Q$ factor is $q_1$.

You can first write a code for $A$ assuming that $A$ is Hessenberg but your final code needs to be for a general matrix $A$.

\item 

We consider $A$ real Hessenberg. We construct $B = A^2 + 4 A - 3 I$. Then perform
the QR factorization of $B$ such that $Q,R = \textmd{qr}(B)$.  Finally
construct $C = Q^T A Q$. Give two explanations of why $C$ is Hessenberg.  One
explanation is based on using $\mathbb{C}$. The other explanation is based on
looking at structure and staying in real arithmetic.

\item {\em Exact shift for pair of complex conjugate eigenvalue.}

We consider $A$ real Hessenberg with a complex conjugate eigenvalue pair $\frac14 \pm i \frac12$.
So $\frac14 \pm i \frac12$ are two eigenvalues of $A$.
We compute 
$B = A^2 - \frac{1}{2} A + \frac{5}{16} I.$ Then $Q,R = \textmd{qr}(B)$. Then $C = Q^T A Q$.
Note that $\frac14 \pm i \frac12$ are roots of the polynomial $x^2 - \frac{1}{2} x + \frac{5}{16}.$

\underline{Question:} What is the structure of the $C$? (From question 2, we know it is Hessenberg, but can we say more if we do an exact shift?)


\end{enumerate}






\end{document}
