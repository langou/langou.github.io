\documentclass[11pt,letterpaper]{article}
\usepackage{amssymb}


\begin{document}

\begin{center}
{\Large \bf Winter 25-26 :: CR 04 :: Weekly assignment \#3 }
{due Tu. 12/16/25, 11:59pm, by email,  }
\end{center}

\small

\begin{enumerate}

\item 
(julia only)
Take a random nonsymmetric $n$-by-$n$ matrix $A$ with real eigenvalues and computes an associated Hessenberg
reduction with
\begin{verbatim}
n = 10
X = randn(n, n)
Lambda = randn(n)
A = ( X * Diagonal(Lambda) ) / X
H = triu(LAPACK.gehrd!(A)[1],-1)
\end{verbatim}
(Note: At this point, it might be a good idea to check that the eigenvalues of $A$ and the eigenvalues of $H$ are the same.)

\begin{enumerate}

\item Write a QR algorithm without shift on $H$ that is $\mathcal{O}(n^2)$ per
iteration. Run the algorithm to completion. (That is to convergence to ``Schur
form''.) Your algorithm needs to compute $Q$ and $T$ such that $A = QTQ^T$, $T$
is triangular, and $Q$ is orthogonal.

\item Now use shifted QR iteration to obtain quadratic convergence.  Use either
the Rayleigh Quotient shift or the Wilkinson shift. (Please implement both
shifts.) Only work on finding one eigenvalue (the lowest rightmost eigenvalue).
Do not run the QR algorithm pass the convergence of the lowest rightmost
eigenvalue. Display (or plot) information to show the quadratic convergence.

\end{enumerate}

\item 
(tex + julia)
Let $A$ be a $n$-by-$n$  nonsymmetric matrix. 
Compute the gradient of the Rayleigh quotient of $A$ at any vector $x$ in $\mathbb{R}^n$.
Compute the gradient of the Rayleigh quotient of $A$ at an eigenvector $x$ of $A$.
(Is the gradient zero or not? If it is not zero, what it is?)
Give your derivation in LaTeX. 
Verify in Julia your formula by doing a Taylor expansion and verifying that your gradient 
gives an approximation of an eigenvalue at the first order. 
(Note: two methods: either do a Taylor approximation or use Calculus.)

\item
(tex + julia)
(GVL - P.7.4.6)
Suppose
$$ A = \left(\begin{array}{cc} w & x \\ y &z \end{array}\right)$$ is a real matrix having $\lambda \pm i \mu$, where $\mu$ is nonzero.
Give an algorithm that stably determines $c = \cos(\theta)$ and $s = \sin(\theta)$ such that
$$ \left(\begin{array}{cc} c & s \\ -s &c \end{array}\right)^T
\left(\begin{array}{cc} w & x \\ y &z \end{array}\right)
\left(\begin{array}{cc} c & s \\ -s &c \end{array}\right) = 
\left(\begin{array}{cc} \lambda & \alpha \\ \beta &\lambda \end{array}\right)
$$ where $\alpha \beta =-\mu^2.$
Verify your work in Julia by proposing an implementation of the computation of $c$ and $s$ given an $A$.
Note that the values of $\alpha$ and $\beta$ are also an output of your computation.
All we know is that $\alpha \beta =-\mu^2.$
Also do not worry too much about the ``stably'' in the question.

\item 
(tex)
(GVL - P.7.4.9)
Show that if $C$ is a companion matrix with distinct eigenvalues $\lambda_1$, $\ldots$, $\lambda_n$, 
then $V C V^{-1} = \textmd{diag}(\lambda_1,\ldots,\lambda_n)$ where
$$ V = \left(\begin{array}{cccc} 
1 & \lambda_1 & \ldots & \lambda_1^{n-1} \\ 
1 & \lambda_2 & \ldots & \lambda_2^{n-1} \\ 
\vdots & \vdots & \ddots & \vdots \\ 
1 & \lambda_n & \ldots & \lambda_n^{n-1} \\
\end{array}\right)
$$

\end{enumerate}






\end{document}
